\documentclass[a4paper,13pt]{article}
\usepackage[T1]{fontenc}
\usepackage{amsmath}
\usepackage[russian]{babel}
\usepackage[utf8]{inputenc} % любая желаемая кодировка
\usepackage{indentfirst} % включить отступ у первого абзаца
\usepackage{geometry} % пакет для установки полей
\geometry{top=2cm} % отступ сверху
\geometry{bottom=2cm} % отступ снизу
\geometry{left=2cm} % отступ справа\
\geometry{right=2cm} % отступ слева
\begin{document} % начало документа
\begin{center} % включить выравнивание по центру
    \large ПОДХОД К РАЗРАБОТКЕ СИСТЕМЫ АВТОМАТИЗИРОВАННОГО УПРАВЛЕНИЯ ТРАНСПОРТНЫМИ ПОТОКАМИ\\
    \large Хействер А.А., Костюк В.П.\\
    \large Саратовский Государственный Технический университет, anton@kheystver.ru
\end{center} % закончить выравнивание по центру
\thispagestyle{empty} % не нумеровать страницу
В последение годы в связи c нарастающими темпами роста количества автотранспорта(личного, общественного, служебного) на дорогах крупных городов актуальной становится проблема управления транспортными потоками. 

В некоторых странах оптимизация транспортных потоков является одной из первостепенных задач городского управления. Например в Токио существует универсальная система управления дорожным движением, которая агрегирует данные от камер наблюдения, автодетекторов, полицейских патрулей и т.п. и на основании полученных данных перераспределяет транспортные потоки~\cite{pa}. На данный момент в России нет прямых аналогов такой системы, поэтому задача управления транспорными потоками является одной из первоочередных.

В общем виде задача управления транспортными потоками формируется следующим образом.

Маршруты пассажирских перевозок

\begin{equation}\label{eq:marshruts}
M = \{ M_i/i= \overline{1,I} \}
\end{equation}

Для каждого маршрута известны остановки

\begin{equation}\label{eq:stops}
\forall M_i :O_j = \{ O^i_{j}/j = \overline{1,J_i} \}
\end{equation}

Регулируемые перекрестки на маршрутах

\begin{equation}\label{eq:crosses}
\forall M_i :U_h = \{ U^i_{h}/h = \overline{1,H_i} \}
\end{equation}

Местоположение транспортных средств на маршрутах, полученные с помощью системы ГЛОНАСС

\begin{equation}\label{eq:objects}
\forall M_i :A_k = \{ A^i_{k}/k = \overline{1,K_i} \}, \forall{i} \in I
\end{equation}

Светофоры и знаки на перекрестках

\begin{equation}\label{eq:crosstime}
\forall U_h^{i} :S^i_{h} = \{ S^{ih}_{\alpha \beta}/\alpha = \overline{1,{\alpha}_h} , \beta = \overline{1,{\beta}_{\alpha}} \}
\end{equation}

Необходимо минимизирвоать время проезда транспртными средствами \begin{math}A_k\end{math} регулируемых перекрестков \begin{math}S^i_{h}\end{math} путем изменения интервалов времени работы светофоров

\begin{equation}\label{eq:crosstimemin}
min \sum_{ikh} S^{ih}_{\alpha \beta^{\prime} k}, \text{где}\ \forall i \in I, \forall k \in K_i, \forall h \in H_j, \beta^{\prime} \in \beta \alpha , \alpha \in \alpha h , \text{где}\ \alpha - \text{индекс светофора}, \beta - \text{параметры светофоров}
\end{equation}

В данном случае в модели задачи (\ref{eq:marshruts})-(\ref{eq:crosstimemin}) транспортные средства для перевозки пассажиров рассматриваются как маркеры транспортных потоков, с использованием которых определяется загруженность дорожных коммуникаций.

Общий алгоритм решения задачи (\ref{eq:marshruts})-(\ref{eq:crosstimemin}) состотоит из следующих этапов:
\begin{enumerate}
\item Формирование геоинформационной системы.
\item Получение ретроспективных данных о положении транспортных средств в каждый конкретный момент времени.
\item Получение данных о временных интервалах работы светофоров.
\item Моделирование загруженности маршрутов
\item Выбор интервалов работы светофоров на перекрестках в данный момент времени.
\end{enumerate}

На первом этапе необходимо построение гео-информационной системы(далее ГИС), описывающей все объекты дорожной инфраструктуры. К обьъектам дорожной инфраструктуры можно отнести дороги, перекрестки, светофоры, знаки и т.п..

На втором этапе проводится ретроспективный сбор данных о положении транспортных средств в каждый конкрутный момент времени. Источником данных может служить бортовая система ГЛОНАСС, установка которой на весь общественый транспорт должна завершиться к 1 июня 2012 года. Полученные данные требуют предварительной обработки и аппроксимации чтобы исключить ошибки, появившиеся в следствии неустойчивости ГЛОНАСС-сигнала.

На третьем этапе необходимо получить даныне о временных интервалах работы светофоров. При этом целесообразно использовать данные, предоставленные ГИБДД.

На четвертом этапе производится моделирование загруженности маршрутов. При этом на основании данных, полученных на втором этапе вычисляется скорости транспортных средств в каждой точке маршрута. Скорость транспортного средства при подъезде к перекрестку зависит от плотности потока и, следовательно, от загруженности перекрестка. Таким образом на основании 

На последнем этапе производится минимизация времени проезда регулируемого перекретска путем решулирвоания временных интервалов работы светофоров. При этом возможно регулирование интервалов в режиме реального времени.

Сложность рассматриваемой задачи (\ref{eq:marshruts})-(\ref{eq:crosstimemin}) обусловлена с одной стороны множественным характером транспортных потоков, регулируемых перекрестков и проезда по ним в соответствие с правилами дорожного движения. С другой стороны при практической реализации такого подхода необходимо решение множества технических и технологических трудностей, связанных как с оснащением общественного транспорта трекерами ГЛОНАСС, так и с установкой регулируемых светофоров и дорожных знаков на оптимизируемых участках дорог.

Решение рассмотренной проблемы потребует значительных затрат материальных ресурсов и времени. Вместе с тем актуальность проблемы требует ее ускоренного решения.

\begin{thebibliography}{1}
\bibitem{pa} 
Дорожный просвет / О. Филина // http://www.kommersant.ru/doc/1815962, 2011.
\end{thebibliography}
\end{document}