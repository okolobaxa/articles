\documentclass[a4paper,13pt]{article}
\usepackage[T1]{fontenc}
\usepackage{amsmath}
\usepackage[russian]{babel}
\usepackage[utf8]{inputenc} % любая желаемая кодировка
\usepackage{indentfirst} % включить отступ у первого абзаца
\usepackage{geometry} % пакет для установки полей
\geometry{top=2cm} % отступ сверху
\geometry{bottom=2cm} % отступ снизу
\geometry{left=2cm} % отступ справа\
\geometry{right=2cm} % отступ слева
\title{Разработка системы информационной поддержки строительной организации. Подсистема управления производством} 
\author{Хействер А.А.}
\date{}
\begin{document} % начало документа
\begin{center} % включить выравнивание по центру
    \large ЗАДАЧА ОПТИМИЗАЦИИ ВРЕМЕННЫХ ИНТЕРВАЛОВ РАБОТЫ СВЕТОФОРОВ В ЗАВИСИМОСТИ ОТ ТЕКУЩЕЙ ДОРОЖНОЙ СИТУАЦИИ\\
    \large Хействер А.А., Костюк В.П.\\
    \large Саратовский Государственный Технический университет, anton@kheystver.ru
\end{center} % закончить выравнивание по центру
\thispagestyle{empty} % не нумеровать страницу
В современном мире остро проблема анализа дорожной ситуации. В настоящее время существует множество различных систем информационной поддержки организаций и предприятий. Все они предоставляют широкие возможности автоматизации бизнес-процессов и снижения издержек за счет контроля над различными сферами производства~\cite{pa}.

Известны множества маршрутов пассажирских перевозок

\begin{equation}\label{eq:marshruts}
M = \{ M_i/i= \overline{1,I} \}
\end{equation}

Для каждого маршрута известны остановки

\begin{equation}\label{eq:stops}
\forall M_i :O_j = \{ O^i_{j}/j = \overline{1,J_i} \}
\end{equation}

Местоположение регулируемых перекрестков на маршрутах

\begin{equation}\label{eq:crosses}
\forall M_i :U_i = \{ O^i_{h}/h = \overline{1,H_i} \}
\end{equation}

Местоположение транспортных средств на маршрутах, полученные с помощью системы ГЛОНАСС

\begin{equation}\label{eq:objects}
\forall M_i :A_i = \{ O^i_{k}/k = \overline{1,K_i} \}, \forall{i} \in I
\end{equation}

Интервалы работы светофоров на перекрестках

\begin{equation}\label{eq:crosstime}
\forall S_h : t^h_{1}, t^h_{2}, $A^\text{где} t^h_{1} - красный сигнал светофора, t^h_{2} - зеленый сигнал светофора
\end{equation}


Большинство систем информационно поддержки предприятий представляю собой комплексные продукты, покрывающие все сферы производства. Это накладывает некоторые минимальные ограничения на стоимость продукта и ограничивает его применение в небольших компаниях, не готовых на покупку дорогих систем информационной поддержки. К тому же такие системы очень сложны в эксплуатации за счет большого объема реализованного функционала, не всегда используемого небольшими предприятиями~\cite{pb}.

В свою очередь существующие небольшие системы не подразумевают лёгкую адаптацию под особенности конкретного производства и не могут полностью решить конкретную задачу по организации информационной поддержки предприятия. Часто в таких системах наоборот  отсутствует функционал, необходимый предприятию.

Данная работа посвящена созданию одной из подсистем информационной системы поддержки строительной организации, а именно подсистемы учета и контроля материалов. При этом реализация подсистемы будет полностью отвечать требованиям отдельно взятого предприятия по организации информационной поддержки его деятельности.

Тема выпускной квалификационной работы:  Разработка системы информационной поддержки строительной организации. Подсистема управления производством.

Цель выпускной квалификационной работы: Повышение эффективности ведения учета материалов в производственных процессах строительного предприятия. 

Практическая значимость работы обусловлена востребованностью её результатов на предприятии ООО Производственно-коммерческая фирма «Пульсар-С» с годовым оборотом более 200 млн. рублей.

Инженерная новизна работы заключается в использовании современных подходов, методов и технологий проектирования информационных систем на конкретном отечественном предприятии.

Пояснительная записка к выпускной квалификационной работе состоят из введения, пяти глав, заключения, библиографического списка и четырёх приложений.

В главе 1 проведен аналитический обзор современных технологий, анализ существующих методик и подходов используемых для построения информационных систем

В главе 2 описан объект и общая задача разработки, техническое задание на систему информационной поддержки, разработана и описана архитектура системы, модели задач и общий алгоритм работы системы.

В главе 3 описана  логическая структура баз данных, рассмотрены основные алгоритмы функционирования системы.

В главе 4 рассмотрена структура комплекса программных средств, разработаны и описаны программные модули, спроектирован интерфейс пользователя,  приведена методика испытания системы, разработаны руководства оператора и программиста.

В заключении изложены основные результаты выпускной квалификационной работы.

В приложениях приведены техническое задание на разработку системы, методика испытаний, руководство оператора и программиста, интерфейс пользователя.
\begin{thebibliography}{2}
\bibitem{pa} 
Использование нейросетевых технологий для распознавания предаварийных ситуаций / Е.А. Зотов, В.П. Костюк // Интернет и инновации: практические вопросы информационного обеспечения инновационной деятельности: материалы Междунар. науч. конф. Саратов: СГТУ, 2008 С. 258-259.

\bibitem{pb} Разработка моделей и методов распознавания аварийных ситуаций при бурении нефтяных и газовых скважин / Е.А. Зотов, В.А. Зварич, А.А. Вислов, В.П. Костюк // Математические методы в технике и технологиях: сб. трудов XXI Междунар. науч. конф.: в 10т. Саратов: СГТУ, 2008. Т.4 С.204-207
\end{thebibliography}

\end{document}