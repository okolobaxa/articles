\documentclass[a4paper,13pt]{article}
\usepackage[T1]{fontenc}
\usepackage[russian]{babel}
\usepackage{amsmath}
\usepackage{cases}
\usepackage[utf8]{inputenc} % любая желаемая кодировка
\usepackage{indentfirst} % включить отступ у первого абзаца
\usepackage{geometry} % пакет для установки полей
\geometry{top=2cm} % отступ сверху
\geometry{bottom=2cm} % отступ снизу
\geometry{left=2cm} % отступ справа\
\geometry{right=2cm} % отступ слева
\begin{document} % начало документа
\begin{center} % включить выравнивание по центру
    \large ПОДХОД К РАЗРАБОТКЕ СИСТЕМЫ ПОДДЕРЖКИ ВЫБОРА МАРШРУТОВ ДВИЖЕНИЯ ГОРОДСКИМ ОБЩЕСТВЕННЫМ ТРАНСПОРТОМ\\
    \large Хействер А.А., Костюк В.П.\\
    \large Саратовский Государственный Технический Университет, anton@kheystver.ru
\end{center} % закончить выравнивание по центру
\thispagestyle{empty} % не нумеровать страницу
Современный человек, житель динамично развивающегося мегаполиса, очень много времени проводит в общественном транспорте (ОТ). Поэтому возникает вопрос о рациональном использовании и минимизации этого времени. В данной работе предлагается один из возможных подходов к минимизации времени, проводимого в ОТ. Эта тема уже затрагивалась в~\cite{kheystver}, но требует дальнейшего развития.

Различные маршруты ОТ движутся с разным графиком, имеют разное количество подвижных средств на маршруте. К тому же на скорость движения ОТ влияют такие параметры, как время суток, дни недели, погодные условия и множество других. Совокупность этих факторов приводит к тому, что даже по одному и тому же участку пути транспортные средства(ТС) могут двигаться с разной скоростью. В этих условиях возникает задача выбора рационального пути, позволяющего добраться в заданный пункт назначения. Современные технологии, такие как GPS и ГЛОНАСС позволяют вести спутниковый мониторинг ОТ в реальном времени. Именно данные спутникового мониторинга могут являться основным источником информации для моделирования~\cite{math}.

В общем виде задача выбора маршрутов ОТ может быть сформулирована следующим образом.

Заданы маршруты ОТ

\begin{equation}\label{eq:marshruts}
M = \{ M_i/i= \overline{1,I} \}, \text{где}
\end{equation}
\begin{equation}\label{eq:stops}
\forall M_i = \{ K_{ij}/j = \overline{1,J_i}, i \in I \}, \forall i \in I, \text{где}\ 
\end{equation}
\begin{equation}\label{eq:points}
K_{ij} = (x^i_j, y^i_j) - \text{координаты точек маршрута (опорные точки, остановки)}
\end{equation}

Опорные точки \begin{math}K_{ij'}\end{math} характеризуют изменение азимута движения по i-му маршруту и используются для фильтрацииисходных данных, полученных со спутников. Число таких точек для каждого маршрута конечно.

Реальные значения времени \begin{math}\tau_{ij}\end{math} нахождения ТС в точках \begin{math}K_{ij}\end{math} на маршрутах \begin{math}M_i\end{math} определяется системами ГЛОНАСС/GPS при различных погодных, дорожных, сезонных и других условиях, \begin{math}\forall i \in I\end{math}, \begin{math}\tau_{ij} \subset T_i\end{math}, где \begin{math}T_i\end{math} - интервал наблюдения.

Необходимо определить

\begin{equation}\label{eq:summ}
\min \sum_{j=1}^{j'}{\tau_{ij}z_{ijj'}}, \text{где}\ z =
\begin{cases}
1,\ \text{если существует начальная j и конечная j' точка i-го маршрута}\\
0,\ \text{в противном случае}
\end{cases}
\end{equation}

Для решения рассматриваемой задачи авторами предлагается подход, в основу которого положено использование нейросетевых моделей (НСМ)~\cite{neiro} для прогноза времени \begin{math}\tau_{ij}\end{math}, \begin{math}\forall i \in I, j \in J_i \end{math}.

Общий алгоритм решения задачи (\ref{eq:marshruts})-(\ref{eq:summ}) состоит из следующих этапов:

\begin{enumerate}
    \item Построение объектно-ориентированного графа Г(K,L), вершинами которого являются точки \begin{math}K_{ij} \in K = {K_{ij}/i=\overline{1,I}, j=\overline{1,J}}\end{math}, а ребрами - взаимосвязи этих точек, относящихся к известным маршрутам, \begin{math}L = \left | l_{ii'jj'} \right |, i,i' \in I, j,j' \in J \end{math}.
    \item Формирование множества НСМ, каждая и которых обносится к соответствующему маршруту
    \item Получение значений \begin{math}\tau_{ij}\end{math} c помощью приемников ГЛОНАСС/GPS данных, \begin{math}\forall i \in I, j \in J_i \end{math}
    \item Прогноз значений \begin{math}\overline{\tau_{ij}}\end{math} на последующий период времени с использованием НСМ
    \item Установка весов вершин \begin{math}K_{ij}\end{math} графа Г в соответствии со значениями \begin{math}\tau_{ij}\end{math}
    \item Выбор рационального маршрута движения между заданными точками \begin{math}K_{ij}\end{math} путем обхода графа Г 
\end{enumerate}

Входные вектора для НСМ имеют одинаковую структуру и формируются с учетом следующих факторов \begin{math}\forall M_i\end{math}:

\begin{itemize}
    \item Временной интервал. Целесообразно обучать систему на короткие временные интервалы, для более качественного прогноза в часы пиковых нагрузок на дорожную сеть
    \item Календарные факторы. К ним можно отнести день недели, тип дня недели (будний, выходной, праздничный предпраздничный)
    \item Тип периода (дачный период, майские или новогодние праздники)
    \item Климатические параметры (температура воздуха, осадки)
    \item Интенсивность движения остального транспорта (данные о пробках)
\end{itemize}

Сложность рассматриваемой задачи (\ref{eq:marshruts})-(\ref{eq:summ}) обусловлена как её динамическим характером, так и множественным характером различного рода нечетких факторов, влияющих на движение ОТ. Во-первых большую сложность представляет собой сбор значений \begin{math}\tau_{ij}\end{math} по причине недоукомплектования ОТ системами спутниковой навигации и закрытым характером доступа к данным мониторинга. Во-вторых значения \begin{math}\tau_{ij}\end{math}, полученные со спутников требуют дополнительной обработки, из-за разного рода искажений, возникающих при отражении сигнала от городских сооружений в условиях плотной застройки. В настоящее время проводятся работы по формированию исходных данных на сайте opensaratov.ru Предлагаемый подход планируется использовать для оптимизации движения пассажиров ОТ в условиях города Саратова. 

\begin{thebibliography}{1}
\bibitem{kheystver} Хействер А.А., Костюк В.П. Подход к разработке системы автоматизированного управления транспортными потоками., Сборник трудов МНК ММТТ-25. Т.9. Саратов: СГТУ, 2011.
\bibitem{math} Введение в математическое моделирование транспортных потоков:учеб. пособие / Гасников А.В., Кленов С.Л., Нурминский Е.А., Холодов Я.А., Шамрай Н.Б. — М.: МФТИ, 2010.
\bibitem{neiro} Уоссермен Ф. Нейрокомпьютерная техника. Теория и практика. — М.: Мир, 1992.
\end{thebibliography}
\end{document}